\documentclass[a4paper,10pt]{article}
\usepackage[utf8]{inputenc}
\usepackage{listings}
\usepackage[portuguese]{babel}
\usepackage{graphicx}
\usepackage{natbib}
\usepackage{subfig}
\title{Uma abordagem MPI para transcrição e tradução de cadeias de DNA}
\author{Erick Grilo, Max Fratane, Elihofni Lima}

\lstset{extendedchars=\true}
\lstset{inputencoding=ansinew}

\begin{document}
\begin{flushright}
\thispagestyle{empty}
\includegraphics[width=.2\textwidth]{IC-UFF.pdf}
\end{flushright}

\begin{center}
\vfill
\vspace{-7em}
\emph{\Large Uma abordagem MPI para transcrição e tradução de DNA}
\begin{flushright}
\vspace{1em}
\makebox[.5\textwidth][l]{\parbox{.5\textwidth}{
\vspace{2em}
Elihofni Lima\\
Erick Grilo\\ 
Max Fratane\\ 
}}
\end{flushright}
\vfill
\end{center}

\newpage
\newpage

\section{Introdução e descrição da aplicação}
\paragraph{}A transcrição do DNA é o processo através do qual o DNA serve de modelo para a síntese de RNA feita por um ser vivo. Apenas uma cadeia de DNA é usada nesse processo, ativada pela enzima RNA-polimerase. Em uma determinada região da molécula de DNA, ocorre a separação das hélices, onde uma delas forma o RNA através do encadeamento de nucleotídeos complementares. Em suma, essa é a fase responsável por parear as bases nitrogenadas do DNA com as do RNA: A do DNA com U do RNA, T do DNA com A do RNA, C do DNA com G do Rna e G do DNA com C do RNA.

\paragraph{}Relembrando alguns conceitos de biologia, A (adenina), C (citosina) T (timina) e G (guanina) são os nucleotídeos que compõem o DNA, e as mesmas, com exceção da timina (que vira U, de uracila, que por sua vez é uma base nitrogenada), compõem o RNA. Cada trinca de 3 dessas bases nitrogenadas é chamada de códons. Um códon codifica um aminoácido, vide \citet{objetivo}.

\paragraph{}Motivado pelo tamanho que uma cadeia de DNA pode ter (em \citet{venter2001sequence}, no mapeamento do genoma humano, por exemplo, foram encontradas aproximadamente 2.91 bilhões de pares de bases nitrogenadas) e pelo processo de transcrição ser uma tarefa repetitiva, a criação de um programa paralelo do tipo SPMD (Simple Program, Multiple Data) aparenta ser uma boa abordagem para a solução desta tarefa.Em  \citet{chibli2008multiprocessor}, por exemplo, uma abordagem usando MPI para solucionar problemas de comparação de \emph{strings}, incluindo a transcrição de DNA para RNA e a identificação de aminoáciods, enquanto em \citet{kleinjung2002parallelized} e \citet{xue2014parallel}, o padrão MPI é utilizado para outro fim, que é o alinhamento de sequências de DNA (que visa encontrar similaridades no DNA que podem indicar relações evolucionárias entre diferentes indivíduos, ou seja, características semelhantes entre indivíduos de espécies diferentes.)\\ 
\paragraph{}O objetivo é paralelizar a transcrição e a tradução de DNA, onde a transcrição é o processo responsável por traduzir uma cadeia de DNA para RNA e a tradução consiste em identificar o aminoácido que o códon em questão representa, a partir de códons de RNA (que foram obtidos a partir da trascrição de códons de DNA para RNA). Note que o processo só se inicia efetivamente quando na cadeia original de DNA dado na entrada é identificado um cístron (uma cadeia de DNA iniciada pelos códons compostos por ATC, ACT ou ATT, que possui informações para a síntese de uma proteína). Enquanto um cístron não é encontrado, não há nenhum processamento efetivo. Quando um cístron é encontrado, o processo se inicia. Como estamos considerando nenhuma mutação genética, o cístron deve ser múltipla de 3 para que o algoritmo funcione corretamente.Já, a cadeia original pode ter qualquer tamanho.\\
\newpage

\section{O método sequencial}

\paragraph{}O método sequencial consiste em tratar toda a cadeia de entrada como uma única cadeia, oriunda da leitura de um arquivo texto onde se encontra tal cadeia. A partir daí, toda a cadeia do DNA é transcrita para RNA da seguinte forma: primeiramente, é identificada o local correto para iniciar a transcrição da cadeia (onde o cístron se encontra); em seguida, a cada três nucleotídeos (um códon), sua transcrição (conversão de DNA para RNA) é feita e a identificação do aminoácido que aquele códon se refere é feita em seguida. Ao término da análise de um códon, o códon DNA original, a sua transcrição para códon RNA e o aminoácido que ele representa é escrito em uma linha do arquivo e esse procedimento se repete por todo o tamanho da cadeia de entrada. Caso o códon não possui 3 nucleotídeos (o tamanho do cístron não é múltiplo de 3), tal codon é ignorado. 

\begin{figure}[!htb]
\includegraphics[scale=0.30]{seq.jpeg}
\caption{Exemplo de uma execução da apicação sequencial, indicando onde o cístron é encontrado na cadeia origina, seu tamanho e o tempo total de execução.}
\label{arvore}
\end{figure}
\newpage


\section{O método paralelo}

\paragraph{}O método paralelo consiste em paralelizar todo o processo de transcrição de DNA em RNA e o de tradução do códon para um aminoácido. Dessa forma, o trabalho é divido em n tarefas (1 tarefa mestre, n-1 tarefas escravo) onde a tarefa mestre é a responsável por repartir a sequência de DNA de entrada para as demais tarefas, ficando com a primeira parte da sequência. Após o envio dessas partes da entrada para cada uma das tarefas escravo, a tarefa mestre faz a transcrição da sua parcela da sequência de DNA para RNA, ao mesmo tempo que as tarefas escravo (após receberem a mensagem que contém os dados da tarefa mestre) também iniciam esse processo. A tarefa mestre então realiza a identificação dos aminoácidos que ela transcreveu e fica aguardando as demais tarefas escravo realizarem a parcela do seu processamento (transcrever e identificar os aminoácidos que lhe foram enviadas).\\
\paragraph{}Quando todas as tarefas escravo terminarem o seu processamento, elas enviam de volta os resultados para a tarefa mestre (os códons transcritos e os aminoácidos identificados), que por sua vez imprime na tela o RNA transcrito e os aminoácidos identificados e escreve em um arquivo tais dados.

\begin{figure}[!htb]
\centering
\subfloat[Início da execução, mostrando o \emph{status} dos processos, tamanho total da cadeia de entrada, tamanho do cístron que será processado, o total de códons, a quantidade de códons enviados para cada processo escravo e parte do resultado final.]{
\includegraphics[scale=0.28]{par1.jpeg}
\label{par1}
}
\quad
\subfloat[Fim da execução, mostrando a confirmação do recebimento dos dados processados pela tarefa mestre, os dados processados (que foram escritos em um arquivo e o tempo total de execução]{
\includegraphics[scale=0.28]{par2.jpeg}
\label{par2}
}
\caption{Exemplo de uma execução da aplicação paralela, no ambiente MPI com dois processos, um mestre e um escravo}
\label{Figura 1}
\end{figure}

\newpage

\section{A implementação paralela}
\paragraph{}O ambiente escolhido para a paralelização da implementação foi o MPI (Message Passing Interface), um padrão de comunicação de dados em computação paralela cujo objetivo busca efetuar a troca de mensagens entre processos (ou threads, dependendo da implementação) de fomra prática e eficiente.\\
\paragraph{} A nossa implementação consiste no uso do ambiente MPI para paralelizar o processo: Ao iniciar o ambiente MPI, o ambiente de execução paralela é inicalizado. Em seguida, as tarefas são criadas e inicializadas no comunicador padrão (MPI\_COMM\_WORLD) e, então, existem duas rotinas possíveis para serm executadas por cada uma das tarefas, que é a rotina destinada para a tarefa mestre (identificada pelo rank 0) e outra para as demais tarefas.\\
\paragraph{}A tarefa mestre então inicia sua execução: primeiro ela efetua a leitura do arquivo (utilizando a função \emph{ler}, disponível em io.c), obtendo a cadeia de DNA inserida como entrada e, em seguida, dividindo a mesma em pedaços menores de sub-cadeias de DNA (por exemplo, para uma cadeia de 60 nucleotídeos, como cada códon possui tamanho 3, nessa etapa a tarefa mestre dividiria a cadeia em 20 sub-cadeias de tamanho 3 cada, utilizando a rotina split, encontrada em transcription.c) e em seguida, envia as subcadeias para as demais tarefas (mantendo a primeira porção para si) por meio da chamada da função MPI\_Send (primeiramente enviando a quantidade de códons que a tarefa irá receber, em seguida enviando os códons), que faz o envio bloqueante síncrono de mensagens (isso significa que a tarefa que está enviando retorna a execução assim que a mensagem foi enviada, o que não implica que a mesma foi recebida pela tarefa receptora).\\
\paragraph{}Após o envio dos dados necessários (quantidade de códons e os códons em si), a tarefa mestre começa a efetuar o processamento na parte dos códons que com ela ficou: primeiro, ela divide a sub-cadeia de DNA que restou em outras sub-cadeias de tamanho 3 (o tamanho de um códon) e, para cada uma dessas sub-cadeias, faz a transcrição de DNA para RNA, utilizando a função transcription (encontrada em transcription.c) e em seguida identificando o aminoácido referente ao códon transcrito, fazendo uso da rotina transcription (que pode ser encontrada em trascription.c).\\
\paragraph{}Nesse momento, as demais tarefas escravo também estão fazendo o mesmo: após o recebimento das mensagens que foram enviadas pela tarefa mestre (recebimento feito por meio da rotina MPI\_Recv), cada uma das tarefas escravo divide a cadeia de DNA recebida em sub-cadeis de tamanho 3 (da mesma forma que a mestre), e, para cada sub-cadeia de DNA gerada a partir da cadeia que a tarefa recebeu, ela faz a transcrição de DNA e a tradução do códon para o aminoácido que ele identifica (da mesma forma que a tarefa mestre). Ao término do processo, elas enviam o resultado do processamento (que seria todos os códons transcritos e cada um dos seus respectivos aminoácidos) para a tarefa mestre\\
\paragraph{} A mestre então, para cada tarefa escravo, ela recebe os dados da tarefa escravo (por meio da rotina MPI\_Recv) e, para cada conjunto de dados recebidos (de cada tarefa), ela vai armazenando as cadeias de RNA transcritos e os seus respectivos aminoácidos. Ao fim da recepção (todas as tarefas escravo concluíram o envio, a tarefa mestre então imprime na tela o resultado do processamento: para cada códon, imprime na tela o códon de DNA original da entrada, a cadeia de RNA transcrita referente à tal códon e o aminoácido que tal cadeia de RNA identifica. Em seguida, escreve o mesmo que ela imprimiu em um arquivo de saída. Ao fim da execução do ambiente MPI, é pego o tempo de finalização (a fim de ver quanto tempo o processamento levou) e o ambiente é finalizado pelo processo main (o mesmo onde o ambiente é inicializado) por meio da rotina MPI\_Finalize.\\

\newpage

\section{Descrição dos experimentos computacionais}
A ideia inicial seria efetuar a medição de desempenho em diversas máquinas. Porém, como não foi possível, a máquina física onde foi testada os códigos é um computador com 8 GB de memória RAM real, processador Intel i7-5500U @ 2.40 GHz, com o S.O Windows 10 64 bits. Os códigos foram efetivamente executados numa máquina virtual usando VirtualBox 5.1.22, onde a máquina virtual possui o S.O Ubuntu 16.04 LTS 64 bits e 4 GB RAM dedicados à ela.

\bibliographystyle{agsm}
\bibliography{bibliografia}
\end{document}
